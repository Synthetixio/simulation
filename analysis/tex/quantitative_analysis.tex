\section{Quantitative System Analysis}

We take the view that falsification is an important aspect of validating
the Havven system. In our quantitative analysis we seek to identify failure modes
of the system, and also to characterise not just \textit{whether} Havven stabilises
nomin prices, but \textit{how much} it does. \\

\noindent It has been observed that analytic methods are often difficult to
apply in the complex and dynamic setting of a market.
One suggested solution to this problem is \textit{agent-based modelling}.
Under this paradigm, we proceed by first defining rational agent behaviour
and then simulating the interplay of those strategies over time.
It's hoped that this can often provide a more effective
method of characterising market behaviour and equilibrium prices,
when analytic reasoning fails.~\cite{poggio2001agent}\\

\noindent Such simulations also provide an immediate means of measuring
quantities of interest. We can discover how varying input parameters
affects system outputs in an experimental fashion, simply by observing
the model.
One important corollary is that this is a way of extracting reasonable
settings for system parameters (such as fee levels) that might be difficult
to reason about \textit{a priori}. These systems, reactive as they are,
also provide a method for testing proposed remedies for any identified failure
modes. \\

\noindent In sum, then, the modelling seeks to answer the following, among
other questions:

\begin{itemize}
    \item Does the system stabilise its nomin price?
    \item Under what conditions can this peg fail?
    \item What are reasonable initial settings for fees and other parameters?
\end{itemize}

\subsection{Modelling Havven}

\paragraph{Environment}

\paragraph{Agents}

\paragraph{Optimal strategy mix}

Please visit research.havven.io for a pre-alpha version of our model.

\pagebreak
