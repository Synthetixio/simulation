\documentclass{article}

\usepackage[activate={true,nocompatibility},final,tracking=true,kerning=true,spacing=true,factor=1100,stretch=10,shrink=10]{microtype}
% activate={true,nocompatibility} - activate protrusion and expansion
% final - enable microtype; use "draft" to disable
% tracking=true, kerning=true, spacing=true - activate these techniques
% factor=1100 - add 10% to the protrusion amount (default is 1000)
% stretch=10, shrink=10 - reduce stretchability/shrinkability (default is 20/20)

\usepackage[utf8]{inputenc}
\usepackage{amssymb}
\usepackage{mathtools}
\usepackage{svg}
\usepackage{graphicx}
\usepackage[superscript, biblabel, nomove]{cite}
\usepackage{hyperref}


\usepackage{draftwatermark}
\SetWatermarkText{DRAFT}
\SetWatermarkScale{3}

\begin{document}

% Macros
\newcommand{\CUR}{\textsc{cur}}
\newcommand{\NOM}{\textsc{nom}}


\title{Havven: a stablecoin system\\ v0.3}
\author{Samuel Brooks, Anton Jurisevic, Kain Warwick}
\date{October 2017}


\begin{figure}
    \centering
    %\includesvg[width=0.33\textwidth]{img/block8logo}
    \includegraphics[width=0.33\textwidth]{img/havvenlogo}
\end{figure}
\maketitle

\begin{abstract}
There is currently no effective decentralised unit of account. Previous attempts to create
stable tokens have either relied on significant centralisation or have been undermined by their
complexity. We present Havven, a representative money system which seeks to achieve price
stability with respect to an external asset. Havven is a dual-token solution, composed of a
stabilised exchange token and the reserve token which backs it. Users are incentivised to maintain
this distributed reserve, and to manage the stable token supply so that it is in proportion with the
value of the collateral. Because the collateral is encapsulated entirely within the system and
distributed among its users, we remove the need for a trusted central authority.
Such a stable cryptocurrency, useful for everyday economic purposes, will accelerate the adoption of
distributed ledger technology.
\end{abstract}

\pagebreak

\section{Introduction}

\subsection{Money and Cryptocurrencies}

There are three primary functions of money; to act as a unit of account, a medium of exchange
and as a store of value. In addition, money should ideally exhibit durability,
portability, divisibility, uniformity, limited supply, and acceptability.
Money has become almost invisible over the past few decades as payment technology has advanced.
Because of this, it is often lost upon users of money that it is itself a technology that can be
improved. Specifically, this means improving the performance of our six desirable properties. \\

\noindent Bitcoin as a technological improvement on existing forms of money is impressive because it manages to
simultaneously improve durability, portability, and divisibility.
Further, it does so without requiring the enforcement of a nation state from which to derive its value.
The Bitcoin supply is, therefore, not subject to control by any central authority.

This fixed monetary policy means that increased adoption has tended to drive the price up over time,
allowing Bitcoin to outperform other forms of money as a store of value, precisely because it is not
subject to debasement and devaluation. Unfortunately this fixed monetary supply creates the potential
for volatility in the short term because there is no mechanism within Bitcoin that can monitor or
adjust to changing demand for the currency.

Thus it has tended to be a poor medium of exchange and an even worse unit of account.
In order for something to perform well as a medium of exchange or unit of account it must remain
relatively stable against other goods and services, because money is ultimately a good that other goods
are denominated in. If the price of money as a good is too varied then it becomes less useful as a
denominator of other goods. \\

\subsection{Stablecoins}

\noindent A stablecoin is a cryptocurrency designed for price stability, such that it can function as
both a medium of exchange and unit of account. It should ideally be as effective for making payments
as fiat currencies like the US Dollar, but still retain the desirable characteristics of Bitcoin;
transaction immutability, censorship resistance and decentralisation. \\

\noindent Cryptocurrencies are in these ways a far better form of money, but have been significantly hindered
in their adoption by the fact that as decentralised systems, they have had relatively inflexible internal
monetary policies. Hence stability continues to be one of the most valuable and yet the most elusive
characteristics.
The fact that we have yet to achieve stability in cryptocurrencies without resorting to extreme
centralisation should by no means be taken as evidence that this problem is insurmountable.
The reality is that the technology to create alternative monetary policies within
cryptoeconomic systems has only existed for a few years. Clearly, significant research into stable monetary
frameworks for cryptocurrencies is required. \\

\subsection{Achieving Stability}

\noindent A viable, autonomous, and decentralised stablecoin is a \textit{sine qua non} for achieving a
decentralised economy.
Such a token is a challenge to design. Central banks have near complete discretion over their money supplies;
a great deal of effort and wealth are expended in supporting the stability of state-backed fiat currencies.
And yet, there are many examples where these efforts have utterly failed.
Stability is far harder in cryptoeconomic systems because their behaviour must be almost completely
defined at the start, which means that there can be very little consideration for events that are unpredicted.
We must likely resign ourselves to the fact that these systems will be only functional under certain conditions.
However, if those conditions are well understood, and the cost of destabilising the system is greater than
any derived benefit, we might ensure that the utility of the system outweighs the risks of participating in it. \\

\noindent With Havven we have accepted that such an approach to building an \textit{intrinsically} stable
currency is likely out of reach in the short term.
If economists cannot agree on the optimal approach to monetary policy, it seems unlikely that a system
constructed with the current orthodoxies baked in could succeed in the long term.
Instead we prefer simply to map the value of a stable token to that of the global reserve currency, the USD.
In this way we provide \textit{nominal} stability, by relying on the stability of the underlying
fiat currency.
This peg, stable as measured with respect to a single external asset, is more achievable, though it has
an obvious limitation: it is only as good as that external asset.
For example, ff the US dollar is subject to hyperinflation, our stable token will be similarly impacted.
However, we do not consider this to be a severe problem, given the historically greater stability of
fiat currencies, and that any other asset can be trivially substituted for USD at any time. \\

\subsection{Havven}
\noindent The mechanism Havven uses for maintaining its peg relies on two linked tokens and a
complex of incentives for stability:

\paragraph{Nomin} The stablecoin itself, whose supply floats. Its price measured in fiat currency should be relatively stable.
Other than price stability, the system should also encourage some adequate level of liquidity for nomins
to act as a useful medium of exchange.

\paragraph{Curit} The collateral token, whose supply is static.
The capitalisation of the curits in the market reflects the system's aggregate value, and the reserve
which backs the stablecoin. Thus, users who hold curits take on the role of maintaining the peg. \\

\noindent Each holder of curits is granted the right to issue a value of nomins in proportion to the USD value
of the curits they hold and are willing to place into escrow. If the user wishes to redeem their escrowed curits, they must
present the system with nomins in order to free their curits and trade them again.
The holders of this token provide both collateral and liquidity, and in so doing assume some
level of risk. To compensate this risk, nomin-issuers will be rewarded with fees the system levies
automatically as part of its normal operation.

In this manner, the system incentivises the issuance and destruction of nomins so that the value of
the nomin pool expands and contracts in proportion with the total value of curits backing them.
If the curit price changes, then the volume of the token pool changes with it.
On the other hand, if the nomin price changes exogenously, then the system is designed to provide
incentives for actors to counteract that change. \\

\subsection{Rationale}

It is clear that the introduction of a new cryptocurrency, in isolation, offers no additional value given
the existing and established alternatives such as Bitcoin and Ethereum. Havven thus seeks to derive value
from the addition of \textbf{stability} to its inherited properties as a modern cryptocurrency.
It is designed to provide a practical medium of exchange, without compromising the benefits that decentralisation
offers in order to substantially improve the technology of money.

There are many applications which Bitcoin's inherently deflationary monetary policy and
volatility presently make impossible. So any token which is able to demonstrate an increment
in utility on these fronts over both fiat and cryptocurrencies will significantly
enhance the uptake of cryptocurrency.

\noindent  In his discussion of Hayek money~\cite{ametrano2016hayek}, Ametrano correctly makes the point that
Bitcoin serves the purpose of crypto-gold much better than it does crypto-unit-of-account due to its volatility
and constrained supply. By contrast, governments -- which mint their own currencies -- can and do execute
discretionary stabilisation policies to manipulate the circulating supply. This kind of powerful lever is not
available to Bitcoin and other supply-constrained currencies of its type, but a similar system whose monetary
policy is algorithmically countercyclical rather than deflationary could inherit the desirable characteristics
of both monetary paradigms. It should be possible, by automatic means, to incentivise the issuance and
destruction of tokens according to demand. In this way, users of such a currency would be allowed to
capitalise it while the system automatically seeks to expand and contract the money supply as its backing
reserve fluctuates in value. By this mechanism we might produce a more perfect currency where supply floats
with necessity, but which is not prone to debasement and other issues commonly associated with
inflationary or deflationary forms of money. Ideally, we also seek to remove some of the distortions created
by traditional monetary policy, which, when it is expansionary, shrinks the purchasing power in every account
which is not a direct beneficiary of that policy.\\

\noindent The Havven stablecoin system is akin to representative money in the sense that the fungible nomin
tokens represent some value held in reserve. We define the curit to be the token of backing value as this is
both the start and end point of using the Havven mechanism; curits develop intrinsic value given their ability
to maintain stability (through nomins) with an external denomination. Hence, nomins have no intrinsic value
because we define curits as carrying the value associated with being able to provide a functioning stable
medium of exchange. \\

\noindent Havven however is not representative money as we have traditionally known it. Historical instantiations,
such as the gold standard which allowed anyone to claim against the reserve, caused exacerbations in times of
economic turmoil. Given that it does not need to act as the primary currency in the market, Havven is relieved
of any pressure to respond and correct for macroeconomic market issues. We leave such manipulations of the money
supply to the whims of central banks, for good or ill. Thus Havven is at its simplest a bridge between fiat and
cryptocurrency, a hybrid of the two technologies and thus for numerous use cases superior to both. But it bears
repeating that whatever monetary policy is applied to the external denomination will flow through to the system.
For example, if the USD is significantly devalued through inflation, so too will the nomin. In this scenario,
the value of curits against the USD will increase and more nomins will be able to be issued against that value,
so long as they are denominated in USD. \\

\noindent The Havven system is designed such that the nomin is both denominated in and
mapped to an external store of value. Throughout this paper we use USD as the reference, however this could
be any external and appropriately fungible asset, such as a commodity or fiat currency. Note that denominations
in other cryptocurrencies are not necessary as these already benefit from the features Havven is implementing
for the external denominator. \\

\pagebreak

\section{Quantitative System Analysis}

We take the view that falsification is an important aspect of validating
the Havven system. In our quantitative analysis we seek to identify failure modes
of the system, and also to characterise not just \textit{whether} Havven stabilises
nomin prices, but \textit{how much} it does. \\

\noindent It has been observed that analytic methods are often difficult to
apply in the complex and dynamic setting of a market.
One suggested solution to this problem is \textit{agent-based modelling}.
Under this paradigm, we proceed by first defining rational agent behaviour
and then simulating the interplay of those strategies over time.
It's hoped that this can often provide a more effective
method of characterising market behaviour and equilibrium prices,
when analytic reasoning fails.~\cite{poggio2001agent}\\

\noindent Such simulations also provide an immediate means of measuring
quantities of interest. We can discover how varying input parameters
affects system outputs in an experimental fashion, simply by observing
the model.
One important corollary is that this is a way of extracting reasonable
settings for system parameters (such as fee levels) that might be difficult
to reason about \textit{a priori}. These systems, reactive as they are,
also provide a method for testing proposed remedies for any identified failure
modes. \\

\noindent In sum, then, the modelling seeks to answer the following, among
other questions:

\begin{itemize}
    \item Does the system stabilise its nomin price?
    \item Under what conditions can this peg fail?
    \item What are reasonable initial settings for fees and other parameters?
\end{itemize}

\subsection{Modelling Havven}

\paragraph{Environment}

\paragraph{Agents}

\paragraph{Optimal strategy mix}

Please visit research.havven.io for a pre-alpha version of our model.

\pagebreak

% 
\section{Qualitative Scenario Analysis}

This section provides a qualitative treatment of the reaction of the Havven system in response to various scenarios listed below:

\begin{enumerate}
	\item Ratio moves favourably
	\item Ratio moves unfavourably
		\subitem Accumulate Curits
			\subsubitem Few Curits available.
			\subsubitem No Curits available.
		\subitem Accumulate Nomis
			\subsubitem Few Nomins available.
			\subsubitem No Nomins available.
	\item Creation of new Curits with new funds (not currently explored).
\end{enumerate}

\pagebreak
\section{Functional description}

Havven works by providing a set of market incentives that support the stability of nomin value with respect to an external asset.

\subsection{Stabilisation mechanisms}

Fundamentally, we wish to configure the system such that it incentivises the desired properties of a stablecoin, namely:
\begin{enumerate}
    \item Price stabilisation
    \item Value transfer
\end{enumerate}

\noindent We focus on value stabilisation as the key enabler for a better form of money; once we have this, we assume that we get value-transfer (market share for the currency) for free. \\

\noindent Let us consider the various ways in which one can maintain a stable price relative to a fiat currency. The question we wish to answer is, ``How can we \textbf{control the supply} of the cryptocurrency such that the price of one unit of the stablecoin matches the price of one unit of the denominating currency?'' This is a challenging scenario because there are multiple related forces at work on the price of each currency. We consider these as two independent groups: ``market forces'' and ``control forces''. \\

\noindent \textbf{Market} forces represent supply and demand. These are necessarily different for each currency, otherwise they would move strictly in unison. \\

\noindent \textbf{Control} forces then are the controls one is able to apply over a currency to affect its value, such as an inflation rate or a buy-back scheme. \\

\noindent Our price mapping then should seek to tune the control forces such that one unit of a control currency equals one unit of the denominating currency. We assume that the forces for one currency are independent of the forces for another. \\

\noindent So what are the mechanisms we can apply to control the value of a currency? We consider:

\begin{itemize}
    \item Issuing new currency to increase supply (inflation)
    \item Buying back existing currency to decrease supply (deflation)
    \item Unilateral balance control (changing account balances to maintain a stable buying power)
\end{itemize}

\noindent Unilateral balance control, such described by Amentrano~\cite{ametrano2016hayek}, is discounted on the basis that an individual's balances being directly modified would be unpalatable to the general population. \\

\noindent This leaves us with simply the forces relating to modifying supply. We will review a number of incentive mechanisms in the design of an economically stable cryptocurrency, including fees, supply control, capital growth, and bonds. This version of the draft whitepaper includes an initial treatment of fees.

\subsection{Investment incentives}

We consider the reasons why any rational actor would buy curits. A potential buyer has at least three avenues for making money in Havven:

\paragraph{Capital gains due to the appreciation of curits:}
Due to its constrained supply, and the intrinsic utility of the stablecoin that it backs, it's reasonable to assume that
curits will appreciate in price.

\paragraph{Interest accrued from fees:}
If the price of curits stabilises for long periods of time, fees may be the only source of revenue. Ideally fees are set at a level where they are both high enough to be an incentive for rent-seekers to hold curits in the long term (thus assuming the risk of providing collateral for the system) and low enough not to be a disincentive for ordinary users to transact in nomins.
It is desirable, perhaps in a future world dominated by micropayments, for these fees to be negligible for end users, while still being macroeconomically important for the system, and for those who capitalise it.

\paragraph{Arbitrage profit:}
It is the arbitrageurs who will ultimately bring the price of nomins back into balance by a triangular circuit through nomins, curits, and the external (crypto or fiat) markets. Arbitrageurs might hold curits for a short time in order to pursue this strategy.



\pagebreak
\subsection{Fees}

There are several key considerations with respect to fee design:

\subsubsection{Fee design considerations}

\paragraph{The purpose of fees}

Fees are intended to be redistributed to actors who support the stability of the system. A fee pool will be distributed periodically for this purpose.
If the system determines that the Nomin price is too low, then fees could be burned. If the price is too high then the system could sell these back
into the system at a discounted rate. The fee collection rate will also be a direct measure of the velocity of money in Havven. It's in the interest
of Curit holders to maximise liquidity in order to maximise their return.

\paragraph{Fee beneficiaries}

One possibility would be simply to award fees to any holder of Curits,
but in this situation holders can get all the benefit without taking any risk.
Although in the aggregate, it would be better for Curits-holders if everyone issued Nomins,
the marginal return for any single player (who cannot issue a large fraction of all circulating Nomins)
of actually issuing them would not outweigh the risk they take on in doing so. If a user can issue 1\% of circulating
Nomins, then doing so will only increase their fee takings by 1\%. Hence rational actors would not be incentivised to issue Nomins at all.
This is a classic tragedy of the commons.

\noindent In order to avoid this situation, we must improve the marginal benefit of issuing Nomins into circulation.
Hence, fees must be paid to those who \textit{issue} Nomins, not just those who hold Curits.

\paragraph{Fee collection}

The system can charge fees whenever any value is transferred, or any state is updated.
\noindent Different fee rates have different macroeconomic effects. We might in general like to set higher Curit than Nomin transfer fees, making the stablecoin itself a lower friction market in order to incentivise its use for exchange. Meanwhile, issuance and redemption fees will change the difficulty of entering and exiting the issuance game. \\

\noindent It is also possible for fees to float. The fee schedule could be altered dynamically in order to stabilise the system. It is even conceivable that the system could set negative fee rates if it needed to and charge punitive fees if a user is above the targeted utilisation ratio. For example, if Nomin liquidity is low, meaning the system wants to incentivise issuance, then Nomin transfer fees could increase, thus having the combined effect of increasing the interest accrued by issuers (thus incentivising issuance) and at the same time making it more expensive to transact in Nomins. This would reduce demand and decrease the liquidity requirements. \\

\noindent Of note, fees are antithetical to arbitrage. The higher the fee, the higher the transaction friction, and the harder it is to make money by arbitrage. For example, if exchange fees amount to 1\% per trade, then a full arbitrage cycle between all three markets, (Nomins, Curits, and fiat) will cost in excess of 3\%. So it would not make sense to undertake arbitrage until such a time as the quoted exchange rate is misvalued by more than 3\% relative to the cross exchange rate. Hence, fees compete with arbitrage to stabilise price. Lower fees allow tighter stabilisation, within a window exactly in proportion with the fee rates themselves.

\subsection{Encouraging liquidity}

\noindent It's desirable that when actors issue Nomins they are actually injected into the liquidity pool for their intended use,
rather than being held by the same actor in order to benefit from both the receipt of fees while retaining the option of using those Nomins to rapidly release their Curits.
In this manner they would accrue fees, but take on none of the risk of spending those Nomins, for they always have an instantaneous option to liquidate their position and escape.
On the other hand, an actor who had done the economically-desirable thing and issued Nomins to the market would be forced to buy them back before redeeming their escrowed curits.

\subsubsection{Non-discretionary Issuance}

One possibility is to simply provide an issuer no control over the tokens they issue. That is, when a quantity nomins is issued, they are generated by the system which then places a sell order at the current going rate for that quantity on an exchange on the behalf of the issuer. When the order is filled, the proceeds in ether are remitted to the issuer. \\

\noindent Conversely, when a quantity of nomins is burned, they must first be obtained from the open market. In this way, a user would indicate an intention to burn, providing sufficient value to buy the proposed quantity of nomins, and the system would bid for that quantity on their behalf, thereby liquidating the user's curit position once the nomins have been obtained. \\

\noindent So one might consider there to be a formal distinction between wallets that issue tokens and those that do not. In this vein, one might envisage an extra fee to be charged to directly transfer nomins (rather than buying from the market) into a wallet that has an outstanding quantity of nomins it has previously issued, but not burnt. The result of this is that it would be less reasonable for an agent to sit on nomins in order to burn them in future as it is more advantageous in times of relative stability to simply buy them from the market. \\

\pagebreak 

\subsection{Price discovery}

One of the key challenges with denominating a cryptocurrency in a fiat currency is the fundamental link this creates to the centralised world; when the denominating currency exists external to the blockchain ecosystem, some bridge must be built so that the system can act with knowledge of the outside world. Often, this is done by sacrificing trust; in order to reclaim system performance, we can trade some of the trustlessness of the design, such as through implementing an trusted ``Oracle'' service in order to gain knowledge of the external world and build a causal link.



\subsection{Utilisation Ratio} Even though rational actor modelling suggests that the price of nomins and curits will equilibrilate
given that an agent may pay up to some multiple of the market value of a nomin in order to release escrowed curits, we are aware that there may be some
prevailing macroeconomic or psychological influences relating to an undercollateralised position
(i.e. if the value of the collateral pool is less than the issued stablecoin). As such, our modelling incorporates the notion of a ``utilisation ratio''
0 < U < 1, such that the system is over-collateralised in an attempt to counteract these potential issues. It may be that resolving an
optimised utilisation ratio is beyond the ability of our agent-based modelling to determine, and as such, selecting this may need to be
informed by the activity of a live system. Thus it is currently intended for Havven to initially include in its governance model the power
to correct the utilisation ratio. This power can be removed over time as the system is proven, perhaps directly linked to some parametric milestones such as nomin velocity and stability.

%\paragraph{Direct Redemption} Direct redemption is a system design option to allow a holder of nomins to redeem any escrowed curits. This option allows more efficient redemption of nomins for the backing value, however introduces the difficulty of liquidating another actor's escrowed curits and interrupting their collection of fees. This can be solved by adding a premium to the price of an escrowed curit (potentially user-defined) over the current value. This premium would need to be paid in ether as to preserve the symmetric issuance and destruction of nomins. \\

\pagebreak

\section{System variables}

\noindent What follows are the main variables of the system. Under each heading, each row will correspond to a single quantity of interest. Each row will have three columns. Leftmost, a mathematical definition of the variable; in the middle, the dimension of the quantity (which units it is measured in); and on the rightmost, a short English summary of the variable.\\

\noindent Certain abbreviations will be used. For example, \(\CUR{}\) and \(\NOM{}\) will be used as abbreviations for Curits and Nomins considered as units of measurement. \\

\paragraph{Prices}
\begin{align*}
    P_c & \ && &(\frac{\text{\$}}{\CUR{}}) && &\text{: curit price.} \\
    P_n & \ && &(\frac{\text{\$}}{\NOM{}}) && &\text{: nomin price.} \\
    \pi &:= \frac{P_c}{P_n} \ && &(\frac{\NOM{}}{\CUR{}}) && &\text{: curit to nomin conversion factor.} \\
    P_c' &= f(V_n, V_v) \cdot R && &(\frac{\text{\$}}{\NOM{} \cdot \text{sec}}) && &\text{: curit price rate of change.}
    \intertext{Here \(R\) is a risk term incorporating, for example, volatility, number of buyers versus sellers, and so on.}
\end{align*}
\\


\paragraph{Money Supply}
\begin{align*}
    &C \ && &(\CUR{}) && &\text{: Quantity of curits, which is constant.} \\
    &C_e \ && &(\CUR{}) && &\text{: Quantity of escrowed curits.} \\
    &N = C_N \cdot \pi \ && &(\NOM{}) && &\text{: Quantity of nomins. This can float.} \\
    &C_N = \frac{N}{\pi} \ && &(\CUR{}) && &\text{: Curit value of issued nomins.}
    \intertext{Ideally, \(C_N \leq C_e\).}
\end{align*}
\\

\paragraph{Utilisation Ratios}
\begin{align*}
    &U = \frac{C_N}{C} \ && &\text{(dimensionless)} && &\text{: Empirical issuance ratio. } \\
    &U_{max} \ && &\text{(dimensionless)} && &\text{: Targeted issuance ratio ceiling.}
    \intertext{Ideally, \(0 \leq U \leq U_{max} \leq 1\), but we need to work out a good level for \(U_{max}\).}
\end{align*}
\\

\paragraph{Microeconomic Variables} These should be defined as functions of \(P_n, \ P_c, \ \text{fees, etc.}\)
\begin{align*}
S_n \ && (\frac{1}{\text{sec}}) && &\text{: average nomin spend rate} \\
S_i \ && (\frac{1}{\text{sec}}) && &\text{: average issuance rate} \\
S_r \ && (\frac{1}{\text{sec}}) && &\text{: average redemption rate}
\end{align*}
\\

\paragraph{Money Movement}
\begin{align*}
    V_n &= S_n \cdot N \ && &(\frac{\NOM{}}{\text{sec}}) && &\text{: nomin transfer rate.} \\
    V_v &= V_i + V_r \ && &(\frac{\CUR{}}{\text{sec}}) && &\text{: nomin} \leftrightarrow \text{curit conversion rate.} \\
    V_i &= (C - C_N) \cdot S_i \ && &(\frac{\CUR{}}{\text{sec}}) && &\text{: nomin issuance rate.} \\
    V_r &= C_N \cdot S_r \ && &(\frac{\CUR{}}{\text{sec}}) && &\text{: curit redemption rate.} \\
    \intertext{\(V_i\) is assumed to grow as there are more free curits in the system. Actually perhaps it should grow with the number of escrowed Curits with no Nomins issued against them.}
    \intertext{\(V_r\), by contrast, is taken to grow proportionally with the number of escrowed Curits.}
\end{align*}
\\

\paragraph{Fees}
\begin{align*}
\intertext{The following fees are ratios, for example 0.1\%, levied on each transaction.}
&F_{nx} & \ && &\text{(dimensionless)} && &\text{: nomin transfer fee} \\
&F_{cx} & \ && &\text{(dimensionless)} && &\text{: curit transfer fee} \\
&F_i & \ && &\text{(dimensionless)} && &\text{: nomin issuance fee} \\
&F_r & \ && &\text{(dimensionless)} && &\text{: curit redemption fee} \\
\intertext{These quantities are the aggregated fees accrued by the system per unit time.}
&Ag_{nx} &:= V_n \cdot F_{nx} \ && &(\frac{\NOM{}}{\text{sec}}) && &\text{: fees taken from nomin transfers.}
\end{align*}

\pagebreak

\section{Alternative approaches}

\subsection{Basecoin}

\paragraph{Description of system}

\noindent Basecoin is described as operating similarly to Havven in that there is separation between a backing token and a transactional token, however Basecoin also separates out a specific ``bond'' token.
The peg to an arbitrary external asset is maintained by using an oracle service to discover the price on an external market, before regulating the supply of ``basecoins'' through actively increasing (issuing new basecoin),
and decreasing (auctioning of bonds) the supply, effectively acting as an autonomous central bank.

\subsubsection{Key issues}

\noindent Basecoin is intended to operate ``as a decentralized, protocol-enforced algorithm, without the need for direct human judgment (sic). For this reason,
Basecoin can be understood as implementing an algorithmic central bank.'' Whilst not without merit, this approach was discarded by Havven due to the high degree
of design complexity required to be anticipated in order to ensure the stabilisation mechanism is effective. The paper states that Monte Carlo simulations have
been run which indicate stability under a range of scenarios, however details are yet to be released by the team. \\

\noindent Another element not explored in the Basecoin whitepaper is the incentives for participants to engage with the cryptoeconomic system itself.
While there is no argument against the utility of stablecoins, there must be incentives inherent in all such systems to ensure the appropriate
participation of all actors. In this case, there are consumers of the stablecoin and active participants in the monetary policy. It is critical
to be able to demonstrate that the incentives within the system will ensure profitable participation strategies for actors. Without this being clarified
it is unclear as to whether there will be uptake by enough users to generate sufficient currency in circulation to support the demand for a stablecoin.
Critically, the removal of Basecoin from the system to ensure the stable peg is predicated on the significant assumption that participants will take positions
in the ongoing bond auctions. This assumption remains untested. \\

\noindent A final point needs to be made with respect to the overarching monetary approach
espoused in the whitepaper. In the section ``Averting Macroeconomic Depressions'' the authors appear to support
money printing and inflationary policies and the subsequent devaluation of currency. Even were it possible to
demonstrate that inflation of the money supply via such a system would be effective in combating a deflationary spiral,
a far better argument could be made that simply by implementing a stable store of value and unit of account that such a
system would not be required. Generally, the apparent assumption that such a system would be achievable and still able to
handle monetary crises in a far future time without centralised intervention stretches credulity. It's not entirely clear
why Basecoin has intended to merely replicate the function of a central bank, rather than aim for pure stability or a
relative-stable approach such as Havven. We are skeptical of any group that would advocate for monetary approaches that are
diametrically opposed to cryptoeconomic efforts to democratise money, and we feel that the proposal to intentionally
create a systematically inflationary monetary system is not the answer. Instead, we should at this point in time be
aiming to construct a system that provides a stable store of value relative to an arbitrary fiat currency. The macroeconomic
benefits of such a system are clear, and for as long as we live in a fiat-dominated world this will continue to be the case.

\subsection{Tether}

\paragraph{Description of system}

Tethers accepts fiat deposits into the Hong Kong-based Tether Limited bank account and issues ``USDT'' (USD Tether) over Bitcoin via the Omni Layer protocol. Tethers are an asset-backed digital token, representing a claim on the cash held in reserve. \\

\noindent The stability of the USDT `coin' effectively relies on the force of external market arbitrage to ensure the peg holds over time.

\paragraph{Key issues}

Despite the whitepaper claiming that the ``goal of any successful cryptocurrency is to completely eliminate the requirement for trust,'' and that each Tether is ``fully redeemable/exchangeable any time for the underlying fiat currency,'' the company's terms of service quite clearly state that ``there is no contractual right or other right or legal claim against us to redeem or exchange your Tethers for money.'' \\

\noindent Tether clearly relies on a manual, centralised proof of existence for the backing asset, and so suffers from the very issue that the Tether whitepaper decries. Indeed the same issue is encountered with tokenised gold, or similarly any other real-world asset where some Oracle bridge is required to interface into a distributed ledger.

\paragraph{Current state}

Recently, Tether announced support for issuing ERC-20 compatible tokens on Ethereum as opposed to releasing ``tethers'' on the Bitcoin blockchain using the Omni Layer protocol. \\

\noindent At the time of writing, the market capitalisation for USDT was approximately \$440m, and the discrepancy regarding their terms of service remains unresolved. \\


% \subsection{MakerDAO}

% \paragraph{Description of system}

% \paragraph{Key issues}

% \paragraph{Current state}

% \subsection{Nubits}

% \paragraph{Description of system}

% \paragraph{Key issues}

% \paragraph{Current state}

\pagebreak

\bibliography{tex/citations}
\bibliographystyle{plain}

\end{document}
